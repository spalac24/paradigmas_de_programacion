\documentclass[a4paper,10pt]{article}
\usepackage[utf8]{inputenc}
\usepackage[spanish]{babel}
\usepackage[margin=2.5cm, left=3cm]{geometry}

%opening
\title{Tarea1}
\author{Santiago Palacio}

\begin{document}

\maketitle

\begin{itemize}
 \item \textbf{Ejemplos de que es un paradigma}
 \begin{itemize}
  \item La fisica clasica es un paradigma, ya que tiene métodos definidos,
puede criticarse a si misma, evoluciona, tiene teorías y conceptos propios de
ella, tiene ejemplares de cómo resolver problemas, y muchas otras propiedades
que caracterizan un paradigma.

  \item Paradigmas de programacion, justifican su nombre dado que tienen los
elementos característicos, historia, teorías y conceptos propios, maneras
clásicas de resolver problemas, relaciones, comunidades, y sobre todo, una gran
evolución dada por revoluciones.

 \end{itemize}

 \item \textbf{¿Qué no es un paradigma?}\\
 La religión, por ejemplo, no es un paradigma. Ya que a pesar de que tiene
teorías, relaciones, historia\ldots no tiene otras características
fundamentales, tales como métodos, ejemplares de problemas, en si misma no se
puede criticar (no pueden haber revoluciones), y la evolución de ella es poca o
nula.

 Un lenguaje de programación, aislado, tampoco puede verse como un Paradigma,
ya que es algo muy concreto, y simplemente es parte de un paradigma que le
engloba.

 De aquí, podemos ver que un paradigma no es simplemente una lógica dominante,
sino que tiene muchos componentes que le hacen ir más allá de eso.

 \item \textbf{Ejemplos de una comunidad Científica}
 \begin{itemize}
  \item Físicos o químicos que, de acuerdo a su época, siguen ideales en los
que toda su comunidad se enfoca. La gran mayoría se enfocan siempre en un solo
paradigma.

  \item Infromáticos, que desde su relativamente reciente surgimiento, han
pasado por múltiples paradigmas. Actualmente alguien que se considere competente
debe
adaptar múltiples paradigmas y usarlos convenientemente.
 \end{itemize}
 
 \item \textbf{¿Qué podemos explorar sobre el tema?}\\
 Otro pensador, Paul Feyerabend, entra en varios desacuerdos con Kuhn. Primero,
en el concepto del desarrollo científico, Feyerabend afirma que no hay 2 
momentos separados de ciencia, normal y pre-paradigmática, sino que los 2
estilos de ciencia conviven siempre simultaneamente. Otro aspecto importante,
es que Feyerabend critica fuertemente el concepto de ciencia normal, ya que
para él esto no es ciencia. Feyerabend dice que la ciencia debe evitar siempre
ser dogmática, el fundamento de la ciencia es crear teorías falsibles y
verificarlas, pero en cuanto se introduce algún ``dogma'', que no es
verificado, entonces se deja de hacer ciencia. 

Feyerabend también habla acerca de los cambios de paradigma, y agrega respecto a
lo que dice Kuhn que el
paradigma actual tiene mucha influencia en el paradigma que vendrá, ya que este
nuevo, si bien modifica muchas cosas, usualmente debe concordar con mucho de lo
que el anterior paradigma decía. Esto se puede ver por ejemplo en la física,
donde cada nuevo paradigma, en vez de ser una reforma completa a lo que antes
se pensaba, es usualmente una reforma, una mejora.

Popper, por otro lado, afirma que los paradigmas deben surgir de una manera 
diferente. Partir de los problemas, postular soluciones tentativas y tratar de 
falsearlas.
Además, de estas soluciones tentativas pueden surgir más preguntas que lleven 
a la creación de nuevos paradigmas.
Esto es, los paradigmas surgen no de los individuos, sino de los problemas.


 \item \textbf{¿Qué tiene que ver este tema con nuestra profesión?}\\
 Como ingenieros de sistemas, a diario nos vemos expuestos a múltiples
paradigmas de los que disponemos para resolver distintos tipos de problemas. Es
nuestra responsabilidad tomar conciencia de ellos, saber qué son, por qué
existen, no simplemente conocerlos superficialmente. Es ahí donde el estudio de
los paradigmas en general se hace relevante.
 

\end{itemize}

\textbf{Bibliografia}
\begin{itemize}
  \item Textos Presentados en clase
    \begin{itemize}
      \item La estructura de las revoluciones científicas, Thomas Kuhn
      \item The paradigms of programming Robert Floyd
      \item La tensión entre ortodoxia y heterodoxia en la creación
científica Antanas Mockus
    \end{itemize}
    
  \item Consulta Externa
  \begin{itemize}
   \item Los condicionamientos sociales en los paradigmas científicos: Popper y
Kuhn
W.R. Daros
http://redalyc.uaemex.mx/pdf/877/87701805.pdf

\item Popper, Khun, Lakatos y Feyerabend Amigos inseparables

http://christiandoyle.files.wordpress.com/2008/03/ensayo2.pdf
  \end{itemize}

\end{itemize}

\end{document}




