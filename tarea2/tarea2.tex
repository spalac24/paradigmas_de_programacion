\documentclass[a4paper,12pt]{article}
\usepackage[utf8]{inputenc}
\usepackage[spanish]{babel}
\usepackage[margin=2.5cm, left=3cm]{geometry}

%opening
\title{Tarea1}
\author{Santiago Palacio}

\begin{document}

\maketitle

Escoger un paradigma de programación y justificar por qué es un paradigma.

\textbf{Programación Funcional}

Vemos que de los elementos que caracterizan a un paradigma, la programación funcional los cumple todos. Por un lado, vemos que tiene una historia característica, que se remonta a los años 50's, con el surgimiento de algunos lenguajes como LISP, Scheme, y posteriormente ML, Haskell, entre otros. Tienen soporte en teorías matemáticas, como el Cálculo Lambda, por lo que tiene muchos teoremas, conceptos, etc. que se derivan de esto. 

También podemos encontrar comunidades que se basan en este paradigma, por lo que cumple con estas características también.

\end{document}




